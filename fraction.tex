%% It is just an empty TeX file.
%% Write your code here.

%大砖变小转
\chapter{分数与小数}
\section{从除法到分数}

在用乘法表计算除法时,如果除不尽,分数是一种自然的解决方案。回忆一下,$13\div4=3R1=3\frac{1}{4}$。这里,$\frac{1}{4}$的意思是,把1分成4份,取其中的一份。所以,分母4告诉我们,这里的计数单位已经不再是个位的计数单位1,更不是十位的计数单位10,而是把1分成4块作为计数单位。

还记得铺砖的例子么,如果每行铺4块砖,只需要铺1行,那么铺完1块砖就完成了$\frac{1}{4}$。如果需要铺2行,那么铺完2块砖也是完成了$\frac{1}{4}$。换句话说,8份中的2份也就相当于4份中的一份。写成数学式子就是$\frac{1}{4}=\frac{2}{8}$。这里面的规律是,如果分子和分母同时乘以一个数,并不改变分数的大小。利用图1中的乘法表,沿着4向上走,可以看到$\frac{1}{4}$和$\frac{2}{8}$, $\frac{3}{12}$, $\frac{4}{16}$都是一样的。


\section{分数的加减}

\subsection{同分母}

\begin{eqnarray}
\frac{1}{4} + \frac{1}{4} & = & \frac{2}{4},\\
\frac{3}{4} - \frac{2}{4} & = & \frac{1}{4}
\end{eqnarray}
    
    这两个式子告诉我们,分数的加减法和整数的加减法类似,但是计数单位变了。计数单位由分母决定,它是不变的,一直是4。具体的有几个计数单位由分子决定,加减法就要使用分子进行运算。我们只要知道(/4)是一个新的计数单位——把1分成4块——就可以理解了。

\subsection{异分母}

\section{小数}



